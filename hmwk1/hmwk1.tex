\documentclass[11pt]{article}
\usepackage{amsmath}
\usepackage{algpseudocode,algorithm}

\setlength{\topmargin}{-0.7in}

\setlength{\textwidth}{6.5in}

\setlength{\oddsidemargin}{0.0in}

\setlength{\textheight}{10.0in}

\setlength{\parindent}{0in}

%\renewcommand\arraystretch{0.0}
\renewcommand\arraystretch{2.4}
%\setlength\minrowclearance{2.4pt}

\renewcommand{\baselinestretch}{1.2}
\newcommand{\problem}[1]{ \medskip \pp $\underline{\rm Problem\ #1}$\\ }

%\newlength{\pagewidth}
%\setlength{\pagewidth}{6.5in}
\pagestyle{empty}

%\def \A{\rm At}
\def\pp{\par\noindent}

\begin{document}

\begin{flushleft}
CSOR W4231.002 -- Spring, 2016
\end{flushleft}

%\begin{flushright}
%Eleni Drinea
%\end{flushright}

%\vspace{0.5in}
 \centerline{\bf Homework 1}
\medskip
\centerline{Shivam Choudhary (sc3973)}
\centerline{Due: 8pm, Monday, February 8, 2015}

\medskip 

\bigskip



\begin{enumerate}

\item(25 points) In the table below, indicate the relationship between functions $f$ and $g$ 
for each pair $(f, g)$  by writing  ``yes" or ``no"  in each box. For example, 
if $f=O(g)$ then write ``yes"  in the first box. Here $\log^b{x} = (\log_2{x})^b$.


\begin{table}[tbh]  
\begin{center}
\begin{tabular}{|c|c|c|c|c|c|c|}\hline  % \label{tab: less}
$f$ & $g$ & \ \ $O$ \ \  & \ \  $o$ \ \   &  \ \ $\Omega$ \ \  &  \ \ $\omega$ \ \ & \ \ $\Theta$  \ \ \\ \hline
$\log^2{n}$ & $25\log{n}$  &no  &no  &yes    & yes & no \\ \hline
$\sqrt{ \log{n} }$ & $(\log{ \log{n}})^4$  & yes  &no  &no    &  yes&no \\ \hline
$3n \log{n}$ & $n\log{3n}$  &  yes &no  &no    &no  &no \\ \hline
$n^{3/5}$ & $\sqrt{n}\log{n}$  & no  &no  &  yes   &yes  &no \\ \hline
$\sqrt{n} + \log{n}$ & $2\sqrt{n}$  & no  &no   &   yes    & no&    no \\ \hline
$n^2 2^n$ & $3^n$  &no  &no  &  yes  &yes  &no \\ \hline
$\sqrt{n} 2^n$ & $2^{n/2+\log{n}}$  &no &no  &yes    &yes  &no \\ \hline
$n \log{3n}$ & $\frac{n^2}{\log{n} }$  &  no &no  &yes    &no  &no \\ \hline
$n!$ & $2^n$  &no  &no &yes    &yes  &no \\ \hline
$\log{n!}$ & $\log{n^n}$  & no  &no  & yes   &no  &no \\ \hline
\end{tabular}
\end{center}
\end{table} 

\newpage
%\item 
%(a) (4 points) Find (with proof) a function $f_1$ such that $f_1(2n)$  is $O(f_1(n))$. 
%(b) (4 points) Find (with proof) a function $f_2$ such that $f_2(2n)$  is not $O(f_2(n))$.


\item (12 points)
Show that, if $\lambda$ is a positive real number, then $f(n)= 1+ \lambda + \lambda^2 + \ldots + \lambda^n$ is
\begin{enumerate}  
\item $\Theta(1)$ if $\lambda<1$. 
\item $\Theta(n)$ if $\lambda=1$.
\item $\Theta(\lambda^n)$ if $\lambda>1$.
\end{enumerate}
\begin{center}
f(n) = $\frac{1 - \lambda^{n+1}}{1-\lambda }$ = $\frac{ \lambda^{n+1}-1 }{\lambda -1 }$ respectively as $\lambda >1$ or $\lambda <1$,when $\lambda$ $\neq 1$ 
\end{center}
(a) $\Theta(1)$ if $\lambda<1$. According to $\Theta{(g(1))} $,
the bounds are defined as c1 $\leq$ f(n) $\leq$ c2
Since 1-$\lambda^{n+1}$ will always be less than 1-$\lambda$,we can always find two constants c1(can be just greater than zero) and c2(can be greater than 1). Hence the sum f(n) is bounded by $\Theta(1).$ 

(b) $\Theta(n)$ if $\lambda=1$. For $\lambda=1$ 
f(n)=n+1 for this case. So $\Theta{(n)} \implies$
c1 n $\leq$ n+1 $\leq$ c2 n.
And this translates to c1 being any quantity $\leq$ 1 and c2 being any quantity $\geq$ 1.

(c) $\Theta(\lambda^n)$ if $\lambda>1$. As $\lambda \geq$ 1, the f(n) can be written as $\lambda^n \leq \lambda^{n+1} -1 \leq \lambda^{n+1}$. Now dividing by $\lambda -1$,we get 
\begin{center}
$\frac {\lambda^n}{\lambda -1} \leq 
\frac {\lambda^{n+1}-1}{\lambda -1} \leq \frac {\lambda^{n+1}}{\lambda -1}$
\end{center}
And hence we can always find two constants c1 and c2 which are in this case given by left most and right most part of the equation.

\bigskip

\item (12 points)
\begin{itemize}
\item Find (with proof) a function $f_1$ such that $f_1(2n)$  is $O(f_1(n))$. 

Let $f_1(n)$ = $\frac{n}{2}$, Now $f_1(2n)$ = n and $O(f_1(n))$ = n. Hence proved
\item Find (with proof) a function $f_2$ such that $f_2(2n)$  is not $O(f_2(n))$.

Let's say f(n) is a recurrence relation. Borrowing and tweaking from Question 5 an example can be f(n) = 8f(n/4)+ $O(n^2)$ which is $O(n^2)$. So f(2n) is 8f(n/2) + O($n^2$) which is O($n^3$). Hence proved.
\item Give a proof or a counterexample: if $f$ is $o(g)$ then $f$ is $O(g)$.

$o(g)$ specifies that the quantity is strictly less than the value whereas $O(g)$ specifies that the quantity is equal or less than the value. This means they both cannot be equal. 

\end{itemize}

\bigskip

\item (24 points) Consider an array $A$ consisting of $n$ integers, $A[1], A[2], \ldots, A[n]$.
You want to output a two-dimensional $n$-by-$n$ array $B$ such that for all
$i<j$, $B[i, j]$ contains the sum of array entries $A[i]$ through $A[j]$, that is,
$A[i]+ A[i+1] + \ldots + A[j]$. (For $i \geq j$, the value of entry $B[i,j]$
is left unspecified, so we don't care what is output for these values.)

The most natural way to solve this problem is provided by the following algorithm. 
\begin{algorithmic}
\State \For{ $i=1, 2, \ldots, n$}
 \For{ $j=i+1, i+2, \ldots, n$}
\State Add up array entries $A[i]$ through $A[j]$
\State Store the result in $B[i,j]$
\EndFor
\EndFor
\end{algorithmic}

\begin{enumerate}
\item (6 points) Let $T(n)$ denote the running time of this algorithm  on an input of size $n$. 
For a function $f$ of your choice, show that $T(n)$ is  $O(f(n))$

\item (6 points) For the same function $f$, show that $T(n)$ is also $\Omega( f(n) )$.  
Hence  $\Theta(f(n))$ is an  asymptotically tight bound for $T(n)$.

\item (12 points) Give a different algorithm for this problem with running time $T'(n)$ 
that is asymptotically better than $T(n)$. In other words, you should show 
that the running time of your new algorithm satisfies $T'(n)= O(g(n))$ for
some function $g(n) = o( f(n) )$.  \\
{\em Hint: The new algorithm should exploit 
the structure of the problem better.}
\end{enumerate}





\newpage
\item (16 points) Give tight asymptotic bounds for the following recurrences.
\begin{itemize}
\item $T(n) = 4T(n/2)  + n^3 -1$.
\item $T(n) = 8T(n/2) + n^2$.
\item $T(n) = 6T(n/3) + n$.
\item  $T(n) = T(\sqrt{n})  + 1$.
\end{itemize}

\bigskip


\item (31 points) The Fibonacci numbers are defined by the recurrence 
\begin{eqnarray}
&& F_0 =  0, F_1=1 \nonumber \\
&& F_n =  F_{n-1} + F_{n-2}  (n \geq 2) \nonumber 
\end{eqnarray}
\begin{enumerate}
\item (4 points) Show that $F_n \geq 2^{n/2}$, $n \geq 6$. 
\item  Assume that the cost of adding, subtracting, or multiplying two integers is $O(1)$, independent 
of the size of the integers.  
\begin{itemize}
\item (4 points) Write pseudocode for an algorithm that computes $F_n$ based on the recursive definition above. 
Develop a recurrence for the running time of your algorithm and 
give an asymptotic lower bound for it.
\item (4 points) Write pseudocode for a non-recursive algorithm that asymptotically performs fewer
additions than the recursive algorithm. Discuss the running time of the new algorithm. 
\item (10 points) Show how to compute $F_n$ in $O(\log{n})$ time 
using only integer additions and multiplications. \\
{\em (Hint: Express $F_n$ in matrix notation
and consider the matrix $${0 \ \ \ 1} \choose {1\ \ \  1} $$
 and its powers.)}
%Provide an upper bound for the running time of this algorithm.
\end{itemize}

\item (9 points) Now assume that adding two $m$-bit integers requires $\Theta(m)$ time and that multiplying 
two $m$-bit integers requires $\Theta( m^2 )$ time. What is the running time of the three
algorithms under this more reasonable cost measure for the elementary arithmetic operations?
\end{enumerate}



\newpage 

\item {\bf (Optional exercise: do NOT return, it will not be graded.)} Solve the following recurrences exactly, and prove that your solutions are correct by induction. 
\begin{itemize}
\item $T(1) = 2, T(n) = T(n-1) + 2n$.
\item $T(1) = 1, T(n) = 2T(n-1) + 1$.
\end{itemize}

\bigskip

\item {\bf (Optional exercise: do NOT return, it will not be graded.)}
Unlike a decreasing geometric series, the sum of the {\em harmonic series} 
$1, \frac{1}{2}, \frac{1}{3}, \frac{1}{4}, \frac{1}{5}, \ldots$ diverges; that is,
$\sum\limits_{i=1}^{\infty} \frac{1}{i} = \infty.$
It turns out that, for large $n$, the sum of the first $n$ terms of this series
can be well approximated as
$\sum\limits_{i=1}^{n} \frac{1}{i} \approx \ln{n} + \gamma$
where $\ln$ is the natural logarithm ($\log$ base $e= 2.718\ldots$) and
$\gamma$ is a particular constant $0.57721\ldots$. 

Show that
$$
\sum_{i=1}^{n} \frac{1}{i} = \Theta(\log{n}).
$$
{\em (Hint: To show an upper bound, decrease each denominator to 
the next power of two. For a lower bound, increase each denominator
to the next power of two.)}
\end{enumerate} 


\end{document}

